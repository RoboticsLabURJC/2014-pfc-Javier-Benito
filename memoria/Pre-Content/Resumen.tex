\chapter*{Resumen}

Con la llegada de las consolas de videojuegos de última generación se ha mejorado enormemente la experiencia de usuario. Uno de los avances más revolucionarios de los últimos años ha sido la aparición de los sensores RGBD, como el Kinect para la consola Xbox. Este sensor permite la interacción con la consola sin contacto físico, simplemente con movimientos del cuerpo.

Debido a las grandes ventajas que aportan este tipo de sensores, se ha usado en otros campos de investigación como la visión artificial y la robótica. La utilización de estos sensores ha permitido un avance en áreas como la autolocalización visual.

El presente proyecto fin de carrera se apoya en este contexto. Se ha diseñado y desarrollado un sistema de odometría visual basado en sensores RGBD. El sistema implementado cuenta con algoritmos capaces de estimar la posicion y la trayectoria 3D del sensor en tiempo real, basándose en la información recogida por éste. El resultado es mostrado en una ventana OpenGL con el mapeado de objetos, la posición de la cámara y su recorrido. Además ha sido validado experimentalmente en un entorno real.

Para el desarrollo del componente se ha empleado el lenguaje de programación C++. Se ha utilizado la plataforma JdeRobot 5.4 y varias librerías externas como ICE para la interfaz de comunicación, OpenCV para el procesado de imágenes, Eigen para los cálculos de álgebra lineal y GTK+ para el desarrollo de la interfaz de usuario, entre otras.