% Chapter Template

\chapter{Introducción} % Main chapter title

\label{Chapter1} % Change X to a consecutive number; for referencing this chapter elsewhere, use \ref{ChapterX}

El ser humano no es consciente del proceso neuronal que tiene lugar en nuestro cerebro con el simple hecho de andar o coger un objeto. Se podría decir que tenemos un super ordenador conectado a los órganos sensoriales capaces de recoger muchísima información y procesarla en un tiempo récord.

Desde la antiguedad ya se estuvo pensando en reproducir las habilidades humanas en algún tipo de máquina, la noción de concebir la mente humana como algún tipo de mecanismo no es reciente es referida en célebres filósofos, sin embargo, no es hasta 1950 y con la noción de la computación cuando se introduce la IA (Inteligencia Artificial) por el científico Alan Turing en su artículo \textit{Maquinaria Computacional e Inteligencia} y donde se empieza a coger interés por este campo que será el precursor de una gran cantidad de desarrollos e innovaciones.

En este proyecto abordaremos el problema de obtener un mapa 3D compacto del
entorno que rodea a nuestro dispositivo a partir de la información aportada por un
sensor RGBD, dando por conocida la información sobre su posición y orientación a
través de algún método externo.

%-----------------------------------
%	SUBSECTION Visión artificial
%-----------------------------------
\section{Visión artificial}

Dentro de la IA, se conoce la visión artificial como el procesado que hace nuestro celebro de las imágenes. Que es actualmente un amplio mundo de investigación y desarrollo. Dispositivos como móviles que usan cámaras (reconocimiento facial).

%-----------------------------------
%	SUBSECTION Odometría visual
%-----------------------------------
\section{Odometría visual}

Entre todas posibilidades de visión artificial este proyecto se va a centrar en la odometría visual, que consiste en ubicar en el espacio el sensor de visión y conocer en qué posición está ubicado o si no saber cuanto y donde se mueve y registrar el movimiento.

%-----------------------------------
%	SUBSECTION Visual SLAM
%-----------------------------------
\section{Visual SLAM}

En el problema conocido como Simultaneous Localization and Mapping (SLAM) busca resolver los problemas que plantea colocar un robot móvil en un entorno y una posición desconocidas, y que él mismo se encuentre capaz de construir incrementalmente un mapa de su entorno consistente y a la vez utilizar dicho mapa para determinar su propia localización.

La solución a este problema conseguiría hacer sistemas de robots completamente autónomos que junto con un mecanismo de navegación el sistema se encontrará con la capacidad para saber a dónde desplazarse, ser capaz de encontrar obstáculos y reaccionar ante ellos de manera inteligente.

La resolución al problema SLAM ha suscitado un gran interés en el campo de la robótica y ha sido resuelto teóricamente de diversas formas como es el caso del artículo \parencite{Reference1}. Y aunque algunas de ellas han obtenido buenos resultados en la práctica siguen surgiendo problemas a la hora de buscar el método más rápido o el que genere un mejor resultado con menos índice de fallo. La búsqueda de algoritmos y métodos que resuelvan estos problemas sigue siendo una tarea pendiente en este ámbito.

