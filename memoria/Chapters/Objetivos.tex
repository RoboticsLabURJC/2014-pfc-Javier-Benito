\chapter{Objetivos}

\label{Chapter2} % Change X to a consecutive number; for referencing this chapter elsewhere, use \ref{ChapterX}

Una vez presentado el contexto, en este capítulo se describen los objetivos concretos que se pretenden alcanzar. Tras una descripción del problema abordado y los requisitos, se detallará la metodología y la planificación que se ha llevado a cabo en la elaboración de este trabajo.

%------------------------------------------------
%	SECTION Descripción del problema
%------------------------------------------------
\section{Descripción del problema}

El objetivo principal de este trabajo es desarrollar un programa que solucione el problema de visualSLAM, para un sensor RGBD, a través de técnicas de odometría visual incrementales. El sistema debe ser capaz de averiguar la posición y orientación 3D de un sensor RGBD que se mueve libremente por el espacio y que va alimentando al sistema con las imágenes obtenidas en tiempo real.

Este objetivo principal se ha articulado en los siguientes cuatro subobjetivos:

\begin{enumerate}
 
\item \underline{Detección de puntos de interés}.
Creación de un módulo encargado de recoger de manera dinámica las imágenes RGB y DEPTH del sensor.

Desarrollo de un componente capaz de recoger, a través de diferentes técnicas, puntos de interés de una imágen con la opción de desarrollar algún filtro para añadir robustez a la detección.

\item \underline{Emparejamiento de puntos}.
Desarrollo del cálculo de correspondencias entre los puntos de interés en el fotograma actual y en el fotograma previo.

\item \underline{Estimación del movimiento 3D}.
Implementación de la matemática necesaria para una vez tenidos los puntos emparejados, entre dos fotogramas consecutivos, calcular la matriz de rotación y traslación (RT) que se necesitará para calcular en ese instante el movimiento incremental del sensor.

\item \underline{Pruebas y experimentos}.
El programa diseñado y construido se validará experimentalmente con un sensor real.

\end{enumerate}

Cada uno de los subobjetivos incluirán sus pruebas unitarias y su interfaz gráfico, desde donde se podrá analizar y verificar cada uno de los hitos.

%------------------------------------------------
%	SECTION Requisitos
%------------------------------------------------
\section{Requisitos}

A parte de las funcionalidades mencionadas en el apartado anterior, la solución final del proyecto debe satisfacer además los siguientes requisitos:

\begin{itemize}

\item Desarrollo del proyecto haciendo uso de la plataforma \textbf{JdeRobot 5.4.0}, que permite abstraerse de algunas de las funcionalidades de más bajo nivel como puede ser la captura de información del sensor o el protocolo de comunicaciones.

\item Como la mayoría de componentes de JdeRobot están en C++, el trabajo también se ha desarrollado utilizando el mismo lenguaje de programación.
%~\ref{ch:Chapter3}.

\item Debe funcionar bajo sistema operativo linux, en este caso Ubuntu 14.04 LTS.

\item Tiene que funcionar con un sensor RGBD real.

\end{itemize}


%------------------------------------------------
%	SECTION Metodología
%------------------------------------------------
\section{Metodología}

Para abordar un proyecto de tal envergadura es necesaria una metodología de desarrollo adecuada para ir progresando de una manera ordenada y efectiva. Se ha optado por el modelo en espiral basado en prototipos propuesto por B. Boehm en 1986 \parencite{Reference7}, ya que permite el desarrollo de una manera progresiva e incremental. 

Esta metodología permite el desarrollo de implementaciones parciales que van siendo probadas a medida que después de cada ciclo se va generando un prototipo más completo que el ciclo anterior. Por lo tanto, en cada ciclo o iteración se va añadiendo complejidad a la vez que se van generando funcionalidades nuevas.

Este modelo, utilizado en este trabajo, ha servido de gran ayuda, ya que permite ir avanzando de menos a más, con unos requisitos dependientes de los anteriores y a la vez diferentes para cada iteración. Además, permite ir evaluando y adaptando la evolución del desarrollo a nuestros intereses, algo que suele ocurrir normalmente en los proyectos de investigación.

En la Figura~\ref{fig:spiral} se puede observar el ciclo completo de desarrollo de software en el modelo en espiral. Cada etapa o ciclo completo está compuesto por cuatro fases:

\begin{itemize}
\item \underline{Identificación de objetivos}. En esta primera fase se deciden y se planifican los objetivos a alcanzar en la siguiente iteración partiendo de lo realizado en el ciclo anterior. En caso de la primera iteración se definen los objetivos iniciales.

\item \underline{Evaluación alternativa}. Aquí se definen requisitos y se estudian las distintas maneras de abordar los objetivos marcados de la etapa anterior. Se estudian los riesgos y se evaluan de qué manera se puedan reducir lo máximo posible. Se debe tener un prototipo antes de la siguiente etapa.

\item \underline{Desarrollo del producto}. En esta fase se diseña y se implementa el producto en base a lo planteado en las anteriores fases. Por último, se verifica y se prueba.

\item \underline{Planificación de la siguiente fase}. Considerando el resultado de la fase anterior, se planifica la siguiente considerando los errores cometidos y los resultados esperados, comenzando así una nueva iteración.

\end{itemize}

\begin{figure}[th]
\centering
\includegraphics[scale=0.62]{Figures/spiral.png}
\decoRule
\caption[Ciclo de vida en espiral]{Ciclo de vida del desarrollo del \textit{software} en el modelo espiral.}
\label{fig:spiral}
\end{figure}

Para las fases de planificación y análisis se han mantenido reuniones semanales con el tutor, con intención de revisar, resolver problemas y encarar los nuevos objetivos establecidos.

A fin de documentar y guardar los hitos realizados en el desarrollo del proyecto, así como los errores cometidos y su posible solución, se ha llevado un seguimiento en mediawiki\footnote{http://jderobot.org/J.benitod-tfg} con los detalles de las diferentes iteraciones, ayudadas a veces por imágenes y/o vídeos.

Para la gestión de código se han usado herramientas \textit{software} de control de versiones, primeramente con Subversion (SVN)\footnote{http://svn.jderobot.org/users/j.benitod/pfc} y finalmente con GIT en un repositorio de GitHub\footnote{https://github.com/RoboticsURJC-students/2014-pfc-Javier-Benito}. Todo el código desarrollado es \textit{software} libre y está accesible a quien lo quiera.

%------------------------------------------------
%	SECTION Planificación
%------------------------------------------------
\section{Planificación del trabajo}

A lo largo del trabajo se han ido proponiendo etapas asesoradas y con supervisión del tutor. Las más importantes son:

\begin{enumerate}

\item \textbf{Familiarización de la herramienta JdeRobot}.

Esta etapa consistió en la instalación y el estudio de la plataforma, profundizando en el uso de algún componente con un objetivo muy concreto y sencillo.

Después, y para entender el funcionamiento de algunos de los componentes a bajo nivel más importantes para el trabajo, se propuso el desarrollo de algunos de ellos en otros lenguajes de programación tales como Java o Python.

\item \textbf{Aprendizaje de las herramientas específicas y técnicas de optimización}.

Aquí, a través de prácticas muy concretas se entendió el funcionamiento de algunas de las librerías esenciales para el proyecto.

\begin{itemize}

\item Se realizó un componente utilizando \textbf{PCL} para el cálculo de planos desde nubes de puntos.

\item Se utilizó \textbf{Eigen} para la resolución mediante optimización de sistemas sobredimensionados de ecuaciones, con descomposición QR y en valores singulares (SVD).

\item \textbf{GSL}, para la resolución de sistemas de ecuaciones, en este caso aplicado a un componente rectificador de imágenes.

\end{itemize}

\item \textbf{Creación de un componente para la extración de puntos de interés y emparejamiento}.

Aquí se empezó a desarrollar el componente final. Esta fase corresponde al desarrollo de la solución a los subobjetivos (1) y (2), en donde se comenzó por la extracción de puntos de interés con SIFT de las imágenes y los diferentes algoritmos para los emparejamientos.

\item \textbf{Registro entre dos nubes de puntos, relacionadas por una rotación, una traslación y ruido}.

En esta fase se propuso una práctica dedicada al subobjetivo (3) de estimación de movimiento incremental, en la que desde una nube de puntos y otra multiplicada por una matriz RT inventada, se sacaría a través de SVD dicha matriz a partir de las nubes de puntos iniciales y con la opción también de añadir ruido gaussiano a una de ellas.

\item \textbf{Integración y pruebas experimentales}.

Esta fue la etapa más larga y costosa. Surgieron multitud de errores que se tuvieron que ir depurando con cada iteración, proponiendo soluciones y alternativas. Fueron necesarias numerosas pruebas para conseguir un desarrollo capaz de aportar una solución estable.

\end{enumerate}